% good ideas for general content layout:
% https://www.reddit.com/r/RPGdesign/comments/kgk19y/anything_obvious_missing/ggfi13i/

%%% 




\chapter{Rolling Dice}

A standard dice roll in Untold Aeons has two "pools" of dice: the skill pool and the difficulty pool. 
For each pool you roll a number of d6 dice equal to your skill level, or difficulty level respectively.
You don't simply add up the values of these dice, but instead you count successes and failures.
If you rolled more successes than failures you succeed, otherwise you fail. 
Successes and failures are counted as follows:

\fancytable{ccc}
{
    \thead{Roll} & \thead{Skill Dice} & \thead{Difficulty Dice}\\
}{
    \textbf{6} & 2 Successes & 2 Failures\\
    \textbf{5} & 1 Success & 1 Failure\\
    \textbf{4} & 1 Success & 1 Failure\\
    \textbf{3} & & \\
    \textbf{2} & & \\
    \textbf{1} & 1 Advantage & 1 Drawback\\
}{}

In addition to this, whenever you roll a 1 on a skill dice you generate an \textit{Advantage}, 
and when you roll a 1 on a difficulty dice you generate a \textit{Drawback}. 
These cancel each other out. 
These don't affect whether you succeed or fail, but they add additional effects to your roll.

% When trying to roll a check with a skill of 0 (either cause you don't have the skill or penalties reduced it to 0):  
% Roll a d6, but you only get a single success on a 6, nothing on 1-5, not even advantage on 1.
\textit{
    Idea: When appropriate, let players roll their skill but without the difficutly dice, and the Narrator rolls the difficulty dice behind the screen. This way the players have some idea of how well they did but are never quite sure.
}

\subsection{Bonuses \& Penalties}
Any situational bonuses simply add extra dice to the "skill pool". 
Similarly, situational penalties add extra dice to the "difficulty pool".


\chapter{Character creation}

\section{Characteristics}

% \begin{itemize}
%     \item Dexterity  -> move speed, dodge in combat, diving for cover, ...
%     \item Toughness  -> HP total, damage reduction, ...
%     \item Willpower  -> Stress limit, Sanity saves, ...
%     \item Intellect  -> starting skills, idea rolls, ...
% \end{itemize}

There are 4 main characteristics that define your character, which all range from 1 to 5. 
Each of them represents an aspect of your character. Sometimes they can be used like regular skills \textit{(e.g. an Intellect roll could be used to solve a puzzle)}, 
but they're mostly used to influence other mechanics. 

To create your character you must distribute 12 points between the 4 characteristics, 
with a minimum of 1 and maximum of 5 for each characteristic.

\begin{description}
    \item[Agility] 
    Your character's physical nimbleness, speed, and general athletic capability.
    It is mostly used for dodging / diving for cover in combat, althletic feats, and your speed in chases.

    \item[Toughness] 
    Your character's physical strength, endurance, and health.
    It determines your Wound limit and Soak, generally making you capable of taking more hits in a fight.

    \item[Willpower]
    Your character's mental fortitide, discipline, drive, and ability to withstand stress.
    It determines your Stress limit, and is used to resist the effects of stress and retain your composure.
    
    \item[Intellect]
    Your character's intelligence, education, and ability to learn and process information.
    Characters with a higher Intellect will have higher initial skills. 
    Intellect also determines your character's general knowledge and can be used as a skill.
\end{description}

% \fancytable{lll}
% {
%     \thead{Stats} & \thead{Used for}\\
% }{
%     \textbf{Agility}   & Athletics, dodging / diving for cover, ...\\
%     \textbf{Toughness} & HP total, damage reduction, ...\\
%     \textbf{Willpower} & Stress limit, Sanity saves, ...\\
%     \textbf{Intellect} & General Knowledge, Idea rolls, ...\\
% }{}

\section{Derived Attributes}

After deciding on your Characteristics you should calculate your derived attributes:

\begin{description}
    % \item[Soak $(Toughness / 2, rounded\ down)$]
    % A higher soak means you'll take less damage from incoming attacks. 

    \item[Wound Limit $(10 + Toughness)$] 
    This represents how much physical damage you can take before you become incapacitated.
    
    \item[Stress Limit $(10 + Willpower)$] 
    Characters with a higher Stress limit are able to withstand more stress before succumbing to it.

\end{description}

% \fancytable{lll}
% {}
% {
%     \textbf{Stress Limit}  & 10 + Willpower\\
%     \textbf{Health}        & 10 + Toughness\\
%     % \textbf{Move Speed}    & 5 + Agility\\
% }{}

% Move speed represents the base "walking" speed of the character while busy fighting or shooting (in yards per round, where a round is about 6 seconds).
% In a full-on sprint characters can go up to 5 times their Move Speed per round, while a sustained run is about 2-3 times their speed per round

\section{Backgrounds}
Your character's background represents how you make a living and what your expertise is.
In more concrete terms, your background determines your starting professional skills you have access to,
but doesn't have much direct influence on the game beyond that. 

\todoi{more fluff and more example backgrounds (With a reference to more later in the book?)}

% \subsection{Example Backgrounds:}

\newcommand{\makebackground}[3]{
    \noindent
    \begin{minipage}{\linewidth}
        % \raggedright
        {\raggedright\normalfont\large\bfseries\scshape #1}\\
        % #2\\#3
        #3
        % \textit{Standard:} #2\\
        % \textit{Professional:} #3
    \end{minipage}
    \par
    % \vspace{\parsep}
}
\vspace{\parskip}
\begin{multicols}{2}
    \makebackground{Artist}{}
    {Art/Craft (Any two), History or Science (Biology), Language (Any)}

    \makebackground{Author}{}
    {History, Science (Literature), Science (Any) or Occult, Language (Any)}

    \makebackground{Criminal}{}
    {Locksmithing, Sleight of Hand, Appraise, Electronics or Heavy Machinery}
    
    \makebackground{Librarian}{}
    {Accounting, Language (Any), Any 2 skills as personal specialties}
    
    \makebackground{Doctor of Medicine}{}
    {Language (Latin), Science (Biology), Medicine, Science (Pharmacy)}

    \makebackground{Occultist}{}
    {Science (Anthropology), Art/Craft (Photography), Occult, Language (Any)}
   
    \makebackground{Private Investigator}{}
    {Art/Craft (Photography), Disguise, Locksmithing, Law}

    \makebackground{Psychologist}{}
    {Psychoanalysis, Language (Any) or Accounting, Science (Pharmacy or Biology), Any skill as personal specialty}

    \makebackground{Professsor / Scientist}{}
    {Language (Any), Any 3 skills as academic specialties}

    \makebackground{Soldier}{}
    {Navigation, Heavy Machinery, Language (Any), Any skill as specialty}

\end{multicols}

\section{Starting Skills}
After distributing points between your primary characteristics and picking a background
you can get started with developing your skills. This is done by spending Experience to improve your skills (as explained below). 

Firstly, add all your background skills to your character sheet with a starting rank of 1 (unless they are already on there). 
% Then you can spend $30 + 4 \times Intellect$ Experience on any of your skills.
Then you can spend 30 Experience, plus an additional $4 \times Intellect$ Experience, on any of your skills.
If you want to add more professional skills on top of your background skills
you will have to pay the cost to learn them.

Note: you are \textit{not} allowed to improve your "Mythos Knowledge" skill during character creation.
This can only be done during play. 

\subsection{Improving Skills}
To improve a skill -- or learn a new skill -- you need to spend Experience.
Experience is typically awarded by the Narrator at the end of every session (or plot arc).

Improving a skill costs Experience equal to the \textit{current} rank in that skill.
For example: \textit{if your Persuasion skill is currently rank 3, then it would cost 3 Experience to increase it to rank 4.}
To learn a new professional skill you have to spend 2 Experience to gain your first rank in your new skill.
You'll have to spend additional Experience to increase it beyond that.
% Learning new professional skills is also possible, as long as you have a source to learn from.
% This could be a teacher (including other players), a book, or any other source of knowledge.
% Then you must spend 2 Experience to gain your first rank in your new skill.


% Afterwards you can spend another $10 + 4 \times Intellect$ on any skill you want as personal interests. 

% When assigning Experience to background skills you \textit{do not} have to pay 
% the Experience cost for learning new professional skills that are part of your background (and they are assumed to start at rank 1),
% however, for personal interest skills you do have to learn any new skills as normal.



\section{Motivations}
\todoi{Talk about character Motivations and guide the player to picking starting motivations.}


\chapter{Skills}
There are two different kinds of skills: standard and professional skills.
Standard skills are all the skills that are on the character sheet by default and start at rank one.
These are skills that everyone knows at least to some extent.

Professional skills on the other hand are skills that only some people know, 
and they aren't listed on the character sheet by default. 
They're usually more specialised and unique skills.

\section{Standard Skills}
Standard skills are skills that every character possesses (at least at a basic level),
and have a starting rank of 1. 
Professional skills are skills that must be learned either in character creation or during play,
therefore these have a starting rank of 0. 

\fancytable[1.5]{m{1in}m{1in}m{1in}}
{
    \thead[l]{General} & \thead[l]{Social} & \thead[l]{Combat}\\
}{
    Driving & Charm & Brawl\\
    First Aid & Deception & Firearms\\
    Investigation & Insight\\
    Perception & Intimidation\\
    Stealth & Persuasion\\
    Streetwise\\
    Survival\\
}{}

\section{Professional Skills}
Professional skills can be just about anything you want, 
but here's a list of some more common ones.
Some skills, like "Science" or "Language", require a specific specialisation to be listed,
which is done in parentheses after the skill.
(These are also marked in the list by having "\textit{(Specific)}" after the skill.)
\textit{E.g. "Science (Biology)"}

\fancytable[1.5]{lll}
{
    % \thead[l]{General} & \thead[l]{Social} & \thead[l]{Combat}\\
}{
    Accounting                      & History                       & Piloting                      \\
    Appraise                        & Language \textit{(Specific)}  & Psychoanalysis                \\
    Animal Handling                 & Law                           & Science \textit{(Specific)}   \\
    Art/Craft \textit{(Specific)}   & Locksmithing                  & Sleight of Hand               \\
    Disguise                        & Medicine                      &                               \\
    Electronics                     & Navigation                    &                               \\
    Heavy Machinery                 & Occult                        &                               \\
}{}


% \begin{multicols}{3}
%     \begin{itemize}[
%         label={},
%         leftmargin=0pt,
%     ]
%         \item Accounting
%         \item Appraise
%         \item Art/Craft (Specific)
%         \item Disguise
%         \item Electronics
%         \item History
%         \item Language (Specific)
%         \item Law
%         \item Locksmithing
%         \item Mechanics
%         \item Medicine
%         \item Navigation
%         \item Occult
%         \item Operating
%         \item Piloting
%         \item Psychoanalysis
%         \item Sleight of Hand
%     \end{itemize}
% \end{multicols}


% \chapter{Motivations}
% During Character creation come up with 2 drives, and one flaw.
% Drives can be people, something you love doing, or even a goal. 
% Just like skills drives have a rank from 1 to 5, indicating how strong they are.

% \noindent\textbf{Basic concept:}
% \ul{
%     \item Start with 2 drives and 1 flaw during character creation.
%     \item \textbf{Drives:}
%     \ul{
%         \item A drive can be a person, something you love doing, a goal, ...
%         \item Drives have a rank from 1 to 5, just like skills
%         \item Drives are mainly used to lessen the effects of stress: %(and for roleplaying)
%         \ul{
%             \item Acting within a drive during downtime increases stress recovery. 
%             (By rolling the drive to recover stress just like you would with Willpower)
%             \item 
%         }
%     }
%     \item 
%     \ul{
%         \item People can use motivations to lessen stress (be it recovering or reducing incoming stress)
%         \item but doing so also causes those things to lower over time
%     }
%     \item Flaws need to bne part of the motivation system, those would be a great place for keepers to use Story points
%     \ul{
%         \item Phobias and manias could also fall under "Flaws", avoiding somewhat problematic terminology
%     }
%     \item (look at mouse guard and fate too for ideas)

% }


\chapterwithsubtitle{Story points}{(i.e. Story / Fate Points)}
% \chapter{Story points}

There are two pools, the Narrator and player pool. 
The Narrator starts with one Story point, and the players pool starts with 1 point for each player.
When a point is used it moves to the other pool. E.g. a players uses a Story point it moves to the Narrator's pool.
This means that using a Story point can give you benefits, but will likely come back to bite you at some point in the future.
  
Whenever someone uses a Story point to get a bonus or re-roll a check 
they should also describe what's actually happening that would change the outcome of the roll.
For example: a player using a Story point to re-roll an Investigation check could say, 
"...".

\subsubsection{Player Uses:}
\begin{itemize}[topsep=0pt]
    \item Reroll dice (sorta like pushing)
    \item deus ex machina: "good thing I brought that shovel!" \par(used at the Narrator's discretion)
    \item +2 on any roll
    \item +2 difficulty to an NPC's roll
    \item Trigger a critical hit in combat. (more details on \autopageref{critical hit})
\end{itemize}

\subsubsection{Narrator Uses:}
\begin{itemize}[topsep=0pt]
    \item Trigger a hallucination/phobia/mania/...
    \item +2 on an NPC's roll
    \item +2 difficulty to a player's roll
    \item I need more here :p
\end{itemize}

% \begin{itemize}
%     \item \textbf{Player Uses:}
%     \begin{itemize}
%         \item Reroll dice (sorta like pushing)
%         \item deus ex machina: "good thing I brought that shovel!" \par(used at the Narrator's discretion)
%         \item +2 on any roll
%         \item +2 difficulty to an NPC's roll
%         \item Trigger a critical hit in combat (more details on \autopageref{critical hit})
%     \end{itemize}
%     \item \textbf{Narrator Uses:}
%     \begin{itemize}
%         \item Trigger a hallucination/phobia/mania/...
%         \item +2 on an NPC's roll
%         \item +2 difficulty to a player's roll
%         \item I need more here :p
%     \end{itemize}
% \end{itemize}

    
\chapter{Stress \& Sanity}

Instead of a Sanity number, characters just have permanent side effects (manias, phobias, ...) when stress goes above the limit. These can then be roleplayed and can also change over time. (e.g. "recovering in a mental ward" as a way to getting rid of them)

\section{Sanity Checks}
When something happens that could cause distress to a character 
the Narrator may ask for a "Sanity Check" to see how they react.
Sanity checks always use the character's Willpower, and the difficulty of the check will be set based on the severity of the cause. 
Any Advantages or Drawbacks rolled will result in a short term effect.

\fancytable[1.5]{lm{3in}}
{
    & \thead{Effects}\\
}{
    \textbf{Success} & Gain Stress equal to the difficulty\\
    \textbf{Failure} & Gain Stress equal to the difficulty plus number of failures (\textit{before} cancelling them out)\\
    \textbf{Advantage} & Adrenaline rush, fearless, bonus on next san check, ...\\
    \textbf{Drawback} & Scream in terror, involuntary action in combat, freeze, ...\\
}{}

If your stress goes above your limit you must make an Willpower roll with 1 difficulty dice for each point you are above your limit.
If you succeed nothing happens, though you must roll again if you gain more stress,
but if you fail you become Shaken.

If your Stress goes above \textit{twice} limit make an Willpower save or go temporarily insane / fall unconsious.
This follows the same pattern as the save for becoming Shaken. So the difficulty is how far you're above your limit (twice the limit in this case), and failing the save makes it happen while succeeding postpones it a bit.


\subsection{Shaken}
\textit{The goal should be to give the players an "oh no..." feeling when they become shaken.}

When you initially become Shaken you suffer from a temporary bout of madness. 
You are free to roleplay this however you like. 
Additionally, the Narrator can also use Story points to trigger 
phobias, manias, delusions, ... for Shaken characters.


While shaken the effects of any Advantages or Drawbacks you roll on a Sanity check become amplified.
Drawbacks might cause you to flee in terror, become catatonic or even gain a phobia or mania.
On the other hand, if you roll an Advantage on a Sanity check with a difficulty \textit{higher}
than your relevant Hardened level your Hardened level increases by one.



% While you're Shaken you suffer the following consequences:
% \begin{itemize}
%     % \item -1 to all rolls (including sanity) until stress is reduced below limit
%     \item The Narrator can use Story points to trigger phobias, manias, delusions, ...
%     \item The effects of any advantages / drawbacks on sanity checks become amplified and more long-term / permanent (this includes the check that put you over the edge, not just the subsequent ones)
%     \begin{itemize}
%         \item \textit{Drawback:} Phobia, catatonia, paranoid, flee in terror, ...
%         \item \textit{Advantage:} Gain a hardened "notch" (explained below)
%     \end{itemize}
% \end{itemize}

\section{Adapting to horror}

Hardened levels represent how used your character is to seeing different kinds of horrors.
There's 3 categories: \textit{Violence}, \textit{Helplessness}, and \textit{Unnatural}.
These levels act like the mental equivalent of armor, reducing the stress "damage" 
you take by your hardened level in the relavant category.

Being hardened also affects and changes the personality of the character, becoming more cold and uncaring for example.


\section{Recovering Stress}
Once a day you make an unopposed Willpower check. 
If your stress is below your limit you lose 1 stress for every success. 
If it is above your limit you lose stress equal to half the number of successes (rounded down).
You always recover at least 1 Stress.

\section*{Goals for sanity mechanics}
\begin{itemize}
    \item Mental Trauma instead of a "Sanity" stat
    \item Harden over time (i.e. stress "armor")
    \item Temporary Positive Effects (Adrenaline rush, ...)
    \item Temporary Negative Effects (Flee, ...)
    \item Stress gain from Sanity checks
    \item Shaken when >= Stress Limit
    \item Go insane / faint when >= 2x Stress Limit
    \item Give players more agency in the outcome of sanity checks? (like Trail of Cthulhu)
\end{itemize}


\chapter{Combat}

\textbf{Goal for combat:} Every roll should be meaningful, not just endless "roll to hit, roll for damage, repeat" 

\section{Wounds}
The health / wounds system is very analogous to the stress system.

If your total damage taken is more than your limit you must make a Toughness roll with 1 difficulty dice for each point you are above your limit.  
If you succeed nothing happens, though you must roll again if you take more damage, if you fail you are Severly Injured.

If your total damage taken goes above \textit{twice} your limit you must make a Toughness save or die. (same mechanic as above)  
You must also make another save every time you take additional damage (including the damage from being Severely Injured).

The Narrator is also always allowed to ask for stress rolls when significant injuries are suffered. 
This is done at the discretion of the Narrator.


\subsection{Severely Injured}
You have suffered a severe injury and are currently incapacitated (be it passed out or simply unable to fight / act)
Additionally, you take one damage every minute (or round?) until your receive medical care


\section{Healing}

\subsection{Medical Care}
A character can receive medical care for wounds either from a First Aid or Medicine skill.
First Aid can only be attempted shorty after the injury was suffered, and can only be attempted once.
The difficulty of the Fist Aid check is based on the severity of the character's wounds 
and how much time has passed since the injury.
A successful First Aid check heals a number of wounds equal to \textit{half} the number of successes rolled (rounded up, and \textit{before} cancellation).

The Medicine skill works very similarly except it takes at least 1 hour to execute,
and it heals for the full number of successes rolled instead of just half.

\subsection{Natural Healing}
Once a day you make an unopposed Toughness check. 
If your injuries are below your limit you heal one wound for every success. 
If they are above your limit you heal a number of wounds equal to half the number of successes (rounded down).
You always heal at least 1 Wound.

\section{Initiative}
There are 2 possible types of turns: Fast and Slow turns.  
In a Slow turn you can take and action \textit{and} move, in a slow turn you can either take an action \textit{or} move.  
At the start of a round everyone involved in the fight chooses if they want to take a Fast or Slow turn.
If multiple combatants go at the same time they can choose the order in which they act.
The side initiating the fight also goes first. So, the turn order is as follows: 

\begin{enumerate}
    \item Initiating side (\textit{Fast})
    \item Defending side (\textit{Fast})
    \item Initiating side (\textit{Slow})
    \item Defending side (\textit{Slow})
\end{enumerate}


\section{Attack Roll}
When being attacked, either:
    - **Dodge**: Roll Dex to avoid, *but* fall prone afterwards
        - Ties are in favour of the defender
        - Defender takes no damage if they win
    - **Defend**: Roll Brawl to defend yourself
        - Ties are in favour of the defender
        - Defender takes no damage if they win
    - **Fight Back**: Roll Brawl to fight back  
        - Ties are in favour of the attacker
        - Defender takes no damage if they win *and* land a hit of their own
    - If both combatants get 0 successes nothing happens 

{
    \small
    \fancytable[1.5]{lcccc}
    {
        \multirow{2}{*}{\thead{Action}} & \multirow{2}{*}{\parbox{0.6in}{\centering \textbf{Defender Skill}}} & \multicolumn{3}{c}{\thead{Outcome}}\\
        \cmidrule(l){3-5}
        & & \tsubhead{Attacker Wins} & \tsubhead{Defender Wins} & \tsubhead{Tie}\\
    }{
        \textbf{Dodge}      & Dex   & Attacker Hits & No Hit\tnote{*} & No Hit\\
        \textbf{Defend}     & Brawl & Attacker Hits & No Hit & No Hit\\
        \textbf{Fight Back} & Brawl & Attacker Hits & Defender Hits & Attacker Hits\\
    }{
        \item[*] The Defender also falls prone, regardless of wether they succeed or not.
    }
}

\subsection{Critical Hits} \label{critical hit}
Critical hits double the amount of damage you deal from your successes 
(but not the base weapon damage).
They can only be triggered by spending a Story point. 


\subsection{Firearms}

Ranged Attack difficulty:
\ul{
    \item Point blank: 1
    \item less than weapon range: 2
    \item less than 2x weapon range: 3
    \item etc...
    \item +1 difficulty when firing at moving target, partial cover, ...
    \item -1 difficulty when firing at ...
}  



The combatant who's being shot at has the option to try diving away 
(only if they're aware they're being shot at). 
If they choose to do so the difficulty of the shot is increased by half the target's Agility (rounding *up*), 
however, the person diving for cover will be prone until they get up again (which takes an action).

Weapon Trait: \textbf{Rapid Fire  <N>} \newline
When you use this weapon you may choose to rapidly fire many shots instead of just one. If you do this you take 1 penalty dice to your Firearms skill. Then, if your attack succeeds, you manage to hit a number of shots up to your MoS (Margin of Success), but never more than \textbf{N}. Each shot deals damage as usual (base damage + \#S)
If you roll advantages you can make a single hit into a critical hit for each advantage you rolled.



\section{Resolving Advantages / Drawbacks}

(See Genesys p.104 for ideas)

Advantage:
    - 1:
        - +1 on next allied skill check
        - notice something useful
        - -1 to target enemy next skill check
    - 2:
        - extra maneuver
        - bypass cover/armor/...
    - 3:
        - ?

Drawback:
  - 1:
      - Suffer 1 stress
  - 2:
      - opponent gets free maneuver
      - +1 to all opponents when targeting this player until next turn
  - 3:
      - ?
  


\section{Damage Calculation}

Damage Dealt = Weapon Base Damage + Successes Rolled (\textit{before} cancelling out)

Damage Taken = Damage Dealt - max(0, Toughness - Piercing) 

If an attack succeeds \textit{with Advantage} it's a critical hit, dealing double base damage (before reducing it by the Toughness)
(Idea: what about double non-base damage? makes more sense sort of, but need to check the maths)

Melee weapons will often deal extra damage based on your Toughness, this is denoted by adding a "+" to the damage number, so a weapons dealing "1+" damage will have a base damage of 1 + your Toughness.

Unarmed attacks have a base damage of "0+", i.e. your Toughness without any bonuses


How to handle shotguns? No clue :p


\chapter{Open Questions \& Feedback}
\ul{
    \item What a bout a RQ-esque Passions mechanic?
    \item There isn't enough damage...
    \ul{
        \item Maybe remove Toughness as armor?
        \ul{
            \item Or use $\lfloor\frac{Toughness}{2}\rfloor$ instead?
            \item This would mess with the Piercing trait too
        }
        \item Or use $5+Toughness$ for wound limit? (could do this for stress too tbh)
        \item Re-think crits / chances for extra high damage
        \ul{
            \item It doesn't really make sense to have crits based on advantages, 
                that sorta goes against the "two axes of success". 
                It'd also mean that crit hits generally have lower damage, 
                cause advantages and successes are inversely-correlated.
            \item Maybe crits when your MoS > some number? Maybe as a weapon trait? 
            \item Or... Story points for crits? This has potential! :p
        }
    }
    \item All Stress saves should be Willpower
    \item \ul{
        \item This does mean we need more things to do with Intellect.
        \ul{
            \item Magic / reading tomes is one option
            \item Idea rolls; maybe make them more mechanically relevant
            \item Honestly, just starting skills, idea rolls, and magic is \textit{probably} enough...
        }
    }
    \item need more Story point uses (especially for the Keeper) and a reason to avoid hoarding them
    \item maybe also remove Knowledge in favour of Intellect
    % \item get rid of the speed stat, it's not needed
    \item Maybe let people roll for sanity with Mythos Knowledge instead of Willpower when it's appropriate
    \item Idea: what about just getting rid of Backgrounds entirely? 
    instead players could just pick 4(?) professional skills + spend xp on anything.
    Simple. Quick. Free!
    \item HYPERLINK ALL THE THINGS!
    \item Rename Science to Knowledge? it's more general and just better I think
}

\clearpage
\section{Playtesters}
Thank you to everyone who's helped me with playtesting and providing valuable feedback!
\ul{
    \item Gabriel Reich %2Boots \textit{(@too\_boots\#8000)}
    \item Wayne \textit{(@Wayne Rossi\#2400)}
    % \item Kyle Scott \textit{(@Lord of the Kyles\#1817)}
    % \item Morgan J \textit{(@Morgan J\#3848)}
    \item ... and others!
}

\clearpage
% mini-environment to temporarily change some layout stuff
{
    %  move the title up a bit more
    \setlength{\beforechapskip}{-3.5\baselineskip-\topskip}
    % and don't print any number stuff or "reserve space" for it
    \renewcommand*{\printchapternonum}{}
    % generate the chapter
    \chapter*{Rules Summary} % the starred version makes it un-numbered
    % increase indent to match number chapters
    \addtocontents{toc}{\setlength{\cftchapterindent}{\cftchapternumwidth}}
    % add it to the ToC manually, cause un-numbered ones don't get added automatically
    \addcontentsline{toc}{chapter}{Rules Summary}
    % and reset the indent back to what it was
    \addtocontents{toc}{\setlength{\cftchapterindent}{0pt}}
}
