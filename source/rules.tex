\chapter{Stats}
\section{Primary Stats}

% \begin{itemize}
%     \item Dexterity  -> move speed, dodge in combat, diving for cover, ...
%     \item Toughness  -> HP total, damage reduction, ...
%     \item Willpower  -> Stress limit, Sanity saves, ...
%     \item Intellect  -> starting skills, idea rolls, ...
% \end{itemize}

There are 4 main characteristics that define your character, which all range from 1 to 5. 
These are the "core" of your character. They can be used like regular skills (e.g. an Intellect roll could be used to solve a puzzle), 
but they're also used to calculate secondary stats and influence other mechanics. 
During character creation you will distribute 12 points between the 4 characteristics.


\begin{center}
    \noindent\begin{tabular}{@{}l l l@{}}
        % \textbullet & \textbf{Dexterity}: & move speed, dodge in combat, diving for cover, ...\\
        \tableheader{Stats} & \tableheader{Used For}\\
        \toprule
        \textbf{Dexterity} & Move speed, dodging / diving for cover, ...\\
        \textbf{Toughness} & HP total, damage reduction, ...\\
        \textbf{Willpower} & Stress limit, Sanity saves, ...\\
        \textbf{Intellect} & Starting skills, Idea rolls, ...\\
        \bottomrule
    \end{tabular}
\end{center}

\section{Secondary Stats}

Secondary stats are stats that are derived from your primary stats.

\begin{center}
    \noindent\begin{tabular}{@{}l l l@{}}
        % & \tableheader{Calculated as}\\
        \toprule
        \textbf{Stress Limit}  & 10 + Willpower\\
        \textbf{Health}        & 10 + Toughness\\
        \textbf{Move Speed}    & 5 + Dexterity\\
        \bottomrule
    \end{tabular}
\end{center}

Move speed represents the base "walking" speed of the character while busy fighting or shooting (in yards per round, where a round is about 6 seconds).
In a full-on sprint characters can go up to 5 times their Move Speed per round, while a sustained run is about 2-3 times their speed per round


\chapter{Skills}

\section{Skill Checks}

The basic dice roll has two "pools" of dice, your skill dice and the difficulty dice. 
You roll a number of d6 dice equal to your skill level (plus or minus and bonuses / penalties), and similarly, you also roll a number of d6 dice equal to the difficulty. You don't add up the values of these dice, but instead you count successes and failures.
If you rolled more successes than failures you succeed, otherwise you fail.

\begin{center}
    % \noindent\begin{tabular}{@{}c c c c c@{}}
    %     \multirow{2}{*}[-3pt]{\textbf{Roll}} & \tableheaderspan{2}{Skill Dice} & \tableheaderspan{2}{Difficulty Dice}\\
    %     \cmidrule(lr){2-3} \cmidrule(lr){4-5}
    %     & \tablesubheader{Success} & \tablesubheader{Advantage} & \tablesubheader{Failure} & \tablesubheader{Drawback}\\
    %     \toprule
    %     \textbf{6} & 2 & 0 & 2 & 0\\
    %     \textbf{5} & 1 & 0 & 1 & 0\\
    %     \textbf{4} & 1 & 0 & 1 & 0\\
    %     \textbf{3} & 0 & 0 & 0 & 0\\
    %     \textbf{2} & 0 & 0 & 0 & 0\\
    %     \textbf{1} & 0 & 1 & 0 & 1\\
    %     \bottomrule
    % \end{tabular}
    \noindent\begin{tabular}{@{}c c c@{}}
        \textbf{Roll} & \textbf{Skill Dice} & \textbf{Difficulty Dice}\\
        \toprule
        \textbf{6} & 2 Successes & 2 Failures\\
        \textbf{5} & 1 Success & 1 Failure\\
        \textbf{4} & 1 Success & 1 Failure\\
        \textbf{3} & & \\
        \textbf{2} & & \\
        \textbf{1} & 1 Advantage & 1 Drawback\\
        \bottomrule
    \end{tabular}
\end{center}

In addition to this, whenever you roll a 1 on a skill dice you generate an "Advantage", and when you roll a 1 on a difficulty dice you generate a "Drawback". These cancel each other out. These don't affect whether you succeed or fail, but they add additional effects to your roll.

When trying to roll a check with a skill of 0 (either cause you don't have the skill or penalties reduced it to 0):  
Roll a d6, but you only get a single success on a 6, nothing on 1-5, not even advantage on 1.

\textit{Idea: When appropriate, let players roll their skill but without the difficutly dice, and the Narrator rolls the difficulty dice behind the screen. This way the players have some idea of how well they did but are never quite sure.
}

\section{List of Skills}

Standard skills are skills that every character possesses (at least at a basic level),
and have a starting value of 1. 
Professional skills are skills that must be learned either in character creation or during play,
therefore these have a starting value of 0. 

\begin{center}
    \noindent\begin{tabular}{@{}m{1in} m{3in}@{}}
        \tableheader{Skills} & \tableheader{Remarks}\\
        \toprule
        \addlinespace[1ex]
        \multicolumn{1}{c}{\textit{Basic}}\\
        \cmidrule(r){1-1}
        Athletics & \\
        Driving & \\
        First Aid & \\
        Investigation & \\
        Knowledge & Represents "basic" knowledge / education level. \\%Can be used instead of any specific knowledge skill, but at a penalty.\\
        Perception & \\
        Stealth & \\
        Streetwise & \\
        Survival & \\
        \addlinespace[1ex]
        \multicolumn{1}{c}{\textit{Social}}\\
        \cmidrule(r){1-1}
        Charm & \\
        Deception & \\
        Insight & \\
        Intimidation & \\
        Persuasion & \\
        \addlinespace[1ex]
        \multicolumn{1}{c}{\textit{Combat}}\\
        \cmidrule(r){1-1}
        Combat & \\
        Firearms & \\
        \addlinespace[1ex]
        \multicolumn{1}{c}{\textit{Languages}}\\
        \cmidrule(r){1-1}
        English & \\


        \bottomrule
    \end{tabular}
\end{center}

% \begin{itemize}
%     \item Basic:
%     \begin{itemize}
%         \item Athletics
%         \item Driving
%         \item First Aid
%         \item Investigation
%         \item Knowledge
%         \begin{itemize}
%             \item Represents "basic" knowledge / education level
%             \item Can be used instead of any specific knowledge skill but with 2 penalty dice
%         \end{itemize}
%         \item Perception
%         \item Stealth
%         \item Streetwise
%         \item Survival
%     \end{itemize}
  
%     \item Social:
%     \begin{itemize}
%         \item Charm
%         \item Deception
%         \item Insight
%         \item Intimidation
%         \item Persuasion
%     \end{itemize}
    
%     \item Combat:
%     \begin{itemize}
%         \item Brawl
%         \item Firearms
%     \end{itemize}
    
%     \item Languages:
%     \begin{itemize}
%         \item English
%     \end{itemize}
% \end{itemize}

  
\section{Character Creation \& Improvement}
The character's background has a set of starting skill levels,
and on top of that players get \_\_\_ starting xp to freely spend.

Players get additional xp based on the Intelligence stat. Or not, it'd be nice to have character XP be the same for everyone. But I do want some way to have Intellect influence how good your statting skills and / or imporvement is.


\begin{center}
    \noindent\begin{tabular}{@{}l l@{}}
        & \tableheader{XP Needed}\\
        \toprule
        \textbf{Improve a skill} & 10 * current rank \\
        & \textit{so getting a skill from 1 to 5 takes 100 xp}\\
        \addlinespace[1ex]
        \textbf{Learn a new skill} & 20 xp (to get level 1)\\
        \addlinespace[1ex]
        \textbf{Improve characteristic?} & maybe 50 * current rank?\\
        \bottomrule
    \end{tabular}
\end{center}


\chapterwithsubtitle{Karma points}{(i.e. Story / Fate Points)}
% \chapter{Karma points}

There are two pools, the Narrator and player pool. 
The Narrator starts with one karma point, and the players pool starts with 1 point for each player.
When a point is used it moves to the other pool. E.g. a players uses a karma point it moves to the Narrator's pool.
This means that using a karma point can give you benefits, but will likely come back to bite you at some point in the future.
  
Whenever someone uses a Karma point to get a bonus or re-roll a check 
they should also describe what's actually happening that would change the outcome of the roll.
For example: a player using a karma point to re-roll an Investigation check could say, 
"...".

  \begin{itemize}
    \item \textbf{Player Uses:}
    \begin{itemize}
        \item Reroll dice (sorta like pushing)
        \item deus ex machina: "good thing I brought that shovel!" \par(used at the Narrator's discretion)
        \item +2 on any roll
        \item +2 difficulty to an NPC's roll
    \end{itemize}
    \item \textbf{Narrator Uses:}
    \begin{itemize}
        \item Trigger a hallucination/phobia/mania/...
        \item +2 on an NPC's roll
        \item +2 difficulty to a player's roll
        \item I need more here :p
    \end{itemize}
\end{itemize}

    
\chapter{Stress \& Sanity}

Instead of a Sanity number, characters just have permanent side effects (manias, phobias, ...) when stress goes above the limit. These can then be roleplayed and can also change over time. (e.g. "recovering in a mental ward" as a way to getting rid of them)

\section{Sanity Checks}
When something happens that could cause distress to a character 
the Narrator may ask for a "Sanity Check" to see how they react.
Sanity checks always use the character's Willpower, and the difficulty of the check will be set based on the severity of the cause. 
Any Advantages or Drawbacks rolled will result in a short term effect.
\begin{center}
    \noindent\begin{tabular}{@{}lm{3in}@{}}
        & \tableheader{Effects}\\
        \toprule
        \textbf{Success} & Gain Stress equal to the difficulty\\
        \addlinespace[1ex]
        \textbf{Failure} & Gain Stress equal to the difficulty plus number of failures (\textit{before} cancelling them out)\\
        \addlinespace[1ex]
        \textbf{Advantage} & Adrenaline rush, fearless, bonus on next san check, ...\\
        \addlinespace[1ex]
        \textbf{Drawback} & scream in terror, involuntary action in combat, freeze, ...\\
        \bottomrule
    \end{tabular}
\end{center}

If your stress goes above your limit you must make an Intellect roll with 1 difficulty dice for each point you are above your limit.
If you succeed nothing happens, though you must roll again if you gain more stress,
but if you fail you become Shaken.

If your Stress goes above \textit{twice} limit make an Intellect save or go temporarily insane / fall unconsious.
This follows the same pattern as the save for becoming Shaken. So the difficulty is how far you're above your limit (twice the limit in this case), and failing the save makes it happen while succeeding postpones it a bit.


\subsection{Shaken}
While you're Shaken you suffer the following consequences:
\begin{itemize}
    \item -1 to all rolls (including sanity) until stress is reduced below limit
    \item Any Drawbacks rolled on skill checks can be used to trigger "fun stuff"! (phobias, manias, delusions, ...)
    \item The goal should be to give the players an "oh no..." feeling when they become shaken
    \item While Shaken, the effects of advantages / drawbacks on sanity checks become amplified and more long-term / permanent (this includes the check that put you over the edge, not just the subsequent ones)
    \begin{itemize}
        \item \textit{Drawback:} Phobia, catatonia, paranoid, flee in terror, ...
        \item \textit{Advantage:} Gain a hardened "notch" (explained below)
    \end{itemize}
\end{itemize}

\section{Recovering Stress}
Once a day you make an unopposed Willpower check. If your stress is below your limit you lose 1 stress for every success. If it is above your limit you lose stress equal to half the number of successes (rounded down) 


\section{Becoming Hardened}

Whenever you roll an advantage on a Sanity check *while Shaken* you gain a "hardened notch" in the respective category (Violence, Helplessness, or Unnatural). When you have three "notches" your hardened level increases by one. 
Being hardened also affects and changes the personality of the character, becoming more cold and uncaring for example.

Hardened levels act like the mental equivalent of armor, reducing the stress "damage" you take by your hardened level in that category.


\section{Goals for sanity mechanics}
\begin{itemize}
    \item Mental Trauma instead of a "Sanity" stat
    \item Harden over time (i.e. stress "armor")
    \item Temporary Positive Effects (Adrenaline rush, ...)
    \item Temporary Negative Effects (Flee, ...)
    \item Stress gain from Sanity checks
    \item Shaken when >= Stress Limit
    \item Go insane / faint when >= 2x Stress Limit
    \item Give players more agency in the outcome of sanity checks? (like Trail of Cthulhu)
\end{itemize}


\chapter{Combat}

\textbf{Goal for combat:} Every roll should be meaningful, not just endless "roll to hit, roll for damage, repeat" 

\section{Wounds}
The health / wounds system is very analogous to the stress system.

If your total damage taken is more than your limit you must make a Toughness roll with 1 difficulty dice for each point you are above your limit.  
If you succeed nothing happens, though you must roll again if you take more damage, if you fail you are Severly Injured.

If your total damage taken goes above \textit{twice} your limit you must make a Toughness save or die. (same mechanic as above)  
You must also make another save every time you take additional damage (including the damage from being Severely Injured).

The Narrator is also always allowed to ask for stress rolls when significant injuries are suffered. This is done at the discretion of the Narrator


\subsection{Severely Injured}
You have likely suffered a permanent injury and are currently incapacitated (be it passed out or simply unable to fight / act)
Additionally, you take one damage every minute (or round?) until your receive medical care


\section{Natural healing}
Once a day you make an unopposed Toughness check. 
If your injuries are below your limit you heal one wound for every success. 
If they are above your limit you heal a number of wounds equal to half the number of successes (rounded down).

\section{Initiative}
There are 2 possible types of turns: Fast and Slow turns.  
In a Slow turn you can take and action \textit{and} move, in a slow turn you can either take an action \textit{or} move.  
At the start of a round everyone involved in the fight chooses if they want to take a Fast or Slow turn.
If multiple combatants go at the same time they can choose the order in which they act.
The side initiating the fight also goes first. So, the turn order is as follows: 

\begin{enumerate}
    \item Initiating side (\textit{Fast})
    \item Defending side (\textit{Fast})
    \item Initiating side (\textit{Slow})
    \item Defending side (\textit{Slow})
\end{enumerate}


\section{Attack Roll}

When being attacked, either:
  - **Dodge**: Roll Dex to avoid, *but* fall prone afterwards
    - Ties are in favour of the defender
    - Defender takes no damage if they win
  - **Defend**: Roll Brawl to defend yourself
    - Ties are in favour of the defender
    - Defender takes no damage if they win
  - **Fight Back**: Roll Brawl to fight back  
    - Ties are in favour of the attacker
    - Defender takes no damage if they win *and* land a hit of their own
  - If both combatants get 0 successes nothing happens 


\section{Firearms}

Ranged Attack:  
Base difficulty: 
  - Point blank: 1
  - less than weapon range: 2
  - less than 2x weapon range: 3
  - etc.

+1 difficulty when firing at moving target, partial cover, ...
-1 difficulty when firing at ...

The combatant who's being shot at has the option to try diving away (only if they're aware they're being shot at). If they choose to do so the difficulty of the shot is increased by half the target's Dexterity (rounding *up*), however, the person diving for cover will be prone until they get up again (which takes an action).

what about auto fire / rapid fire?  

Weapon Trait: \textbf{Rapid Fire  <N>}
When you use this weapon you may choose to rapidly fire many shots instead of just one. If you do this you take 1 penalty dice to your Firearms skill. Then, if your attack succeeds, you manage to hit a number of shots up to your MoS (Margin of Success), but never more than \textbf{N}. Each shot deals damage as usual (base damage + \#S)
If you roll advantages you can make a single hit into a critical hit for each advantage you rolled.



\section{Resolving Advantages / Drawbacks}

(See Genesys p.104 for ideas)

Advantage:
  - 1:
    - +1 on next allied skill check
    - notice something useful
    - -1 to target enemy next skill check
  - 2:
    - extra maneuver
    - bypass cover/armor/...
  - 3:
    - ?

Drawback:
  - 1:
    - Suffer 1 stress
  - 2:
    - opponent gets free maneuver
    - +1 to all opponents when targeting this player until next turn
  - 3:
    - ?
  


\section{Damage Calculation}

Damage Dealt = Weapon Base Damage + Successes Rolled (*before* cancelling out)

Damage Taken = Damage Dealt - max(0, Toughness - Piercing) 

If an attack succeeds *with Advantage* it's a critical hit, dealing double base damage (before reducing it by the Toughness)
(Idea: what about double non-base damage? makes more sense sort of, but need to check the maths)

Melee weapons will often deal extra damage based on your Toughness, this is denoted by adding a "+" to the damage number, so a weapons dealing "1+" damage will have a base damage of 1 + your Toughness.

Unarmed attacks have a base damage of "0+", i.e. your Toughness without any bonuses


How to handle shotguns? No clue :p


\chapter{Open Questions \& Feedback}
\ul{
    \item What a bout a RQ-esque Passions mechanic?
    \item There isn't enough damage\dots
    \ul{
        \item Maybe remove Toughness as armor?
        \ul{
            \item Or use $\lfloor\frac{Toughness}{2}\rfloor$ instead?
        }
        \item Re-think crits / chances for extra high damage
        \item something else?
    }
}
